%-------------------------------------------------------------------------------
% CONFIGURATIONS
%-------------------------------------------------------------------------------
% A4 paper size by default, use 'letterpaper' for US letter
\documentclass[11pt, a4paper]{russell}

% Configure page margins with geometry
\geometry{left=1.5cm, top=1.5cm, right=1.5cm, bottom=2cm, footskip=.5cm}

% Specify the location of the included fonts
\fontdir[fonts/]

% Color for highlights
% russell Colors: russell-emerald, russell-skyblue, russell-red, russell-pink, russell-orange
%                 russell-nephritis, russell-concrete, russell-darknight, russell-purple
\colorlet{russell}{russell-black}
% Uncomment if you would like to specify your own color
% \definecolor{russell}{HTML}{CA63A8}

% Colors for text
% Uncomment if you would like to specify your own color
% \definecolor{darktext}{HTML}{414141}
% \definecolor{text}{HTML}{333333}
% \definecolor{graytext}{HTML}{5D5D5D}
% % \definecolor{lighttext}{HTML}{999999}

\definecolor{darktext}{HTML}{000000}
\definecolor{text}{HTML}{000000}
\definecolor{graytext}{HTML}{414141}
\definecolor{lighttext}{HTML}{414141}

% Set false if you don't want to highlight section with russell color
\setbool{acvSectionColorHighlight}{true}

% If you would like to change the social information separator from a pipe (|) to something else
\renewcommand{\acvHeaderSocialSep}{\quad\textbar\quad}


%-------------------------------------------------------------------------------
%	PERSONAL INFORMATION
%	Comment any of the lines below if they are not required
%-------------------------------------------------------------------------------
% Available options: circle|rectangle,edge/noedge,left/right
% \photo[rectangle,edge,right]{./examples/profile}
\name{Arturo}{Merino}
% \position{Software Architect{\enskip\cdotp\enskip}Security Expert}
% \address{Office 425, Saarland Informatics Campus E1.3, 66123 Saarbrücken, Germany}

% \mobile{+56-}
\email{arturo.merino@uoh.cl}
%\dateofbirth{January 1st, 1970}
\homepage{amerino.cl} 
% \github{github.com/themagicalmammal}
% \linkedin{linkedin.com/in/themagicalmammal}
% \gitlab{gitlab-id}
% \stackoverflow{SO-id}{SO-name}
% \twitter{@twit}
% \skype{skype-id}
% \reddit{reddit-id}
% \medium{madium-id}
% \kaggle{kaggle-id}
\googlescholar{-EnYzW0AAAAJ}{Google Scholar}
% \orcid{0000-0002-1728-6936}
\dblp{239/8813}
\arxiv{merino\_a\_1}
%% \firstname and \lastname will be used
% \googlescholar{googlescholar-id}{}
% \extrainfo{extra information}

% \quote{``Simplicity is deceptively complicated."}


%-------------------------------------------------------------------------------
\begin{document}

% Print the header with above personal informations
% Give optional argument to change alignment(C: center, L: left, R: right)
\makecvheader

% Print the footer with 3 arguments(<left>, <center>, <right>)
% Leave any of these blank if they are not needed
\makecvfooter
  {\today}
  {}
  {\thepage}


%-------------------------------------------------------------------------------
%	CV/RESUME CONTENT
%	Each section is imported separately, open each file in turn to modify content
%-------------------------------------------------------------------------------
<<<<<<< HEAD
% %-------------------------------------------------------------------------------
%	SECTION TITLE
%-------------------------------------------------------------------------------
\cvsection{Personal Information}
%-------------------------------------------------------------------------------
%	CONTENT
%-------------------------------------------------------------------------------
\begin{cvskills}

%---------------------------------------------------------
  \cvskill
    {Full name} % Category
    {Arturo Ignacio Merino Figueroa} % Skills

%---------------------------------------------------------
  \cvskill
    {D.O.B.} % Category
    {24/08/1993} % Skills

%---------------------------------------------------------
\cvskill
{P.O.B.} % Category
{Ñuñoa, Santiago, Chile} % Skills

%---------------------------------------------------------
  \cvskill
    {Citizenship} % Category
    {Chilean} % Skills
    
\end{cvskills}

% %-------------------------------------------------------------------------------
%	SECTION TITLE
%-------------------------------------------------------------------------------
\cvsection{Professional Experience}


%-------------------------------------------------------------------------------
%	CONTENT
%-------------------------------------------------------------------------------
\begin{cventries}

%---------------------------------------------------------
\cventry
{Assistant Professor} % Job title
{Universidad de O'Higgins} % Organization
{Rancagua, Chile} % Location
{2023 - Current} % Date(s)
{
  \begin{cvitems}
    \item Assistant professor at the Engineering Institute of the Universidad de O'Higgins. 
  \end{cvitems}
}

\cventry
{Postdoctoral Researcher} % Job title
{Universität des Saarlandes and Max Planck Institut für Informatik} % Organization
{Saarbrücken, Germany} % Location
{2023-2024} % Date(s)
{
  \begin{cvitems}
    \item Researcher within the algorithms group.
  \end{cvitems}
}
\cventry
    {Project Engineer} % Job title
    {Center for Mathematical Modeling - Resource management laboratory} % Organization
    {Santiago, Chile} % Location
    {2018-2019} % Date(s)
    {
    \begin{cvitems}
      \item Designer and developer of heuristics and algorithms for vehicle routing problems.
    \end{cvitems}
    }
%---------------------------------------------------------
\end{cventries}

% %-------------------------------------------------------------------------------
%	SECTION TITLE
%-------------------------------------------------------------------------------
\cvsection{Education}


%-------------------------------------------------------------------------------
%	CONTENT
%-------------------------------------------------------------------------------
\begin{cventries}

  \cventry
    {Ph.D. in Mathematics} % Degree
    {Technische Universität Berlin} % Institution
    {Berlin, Germany} % Location
    {2019 - 2023} % Date(s)
    {%
    \begin{cvitems}
      \item Graduated summa cum laude.
      Supervised by Prof. Torsten Mütze.
      \item[] Thesis title: ``Combinatorial Generation: Greedy Approaches and Symmetry.''
      \item[] Awardee of the 2023 MATH+ Dissertation award.
      \item[] DOI: \href{https://doi.org/10.14279/depositonce-19653}{10.14279/depositonce-19653}.
    \end{cvitems}
}
  
  \cventry
    {M.Eng. in Applied Mathematics} % Degree
    {Universidad de Chile} % Institution
    {Santiago, Chile} % Location
    {2017 - 2018} % Date(s)
    {
      \begin{cvitems}
      \item Graduated with highest distinction (6.8/7.0). Supervised by Prof. José Soto. 
      \item[] Thesis title: ``Optimal Bases of Uncertainty Matroids and How to Compute Them With Queries of Minimum Cost.''
      \item[] Available at Universidad de Chile's repository: \href{https://repositorio.uchile.cl/handle/2250/168154}{repositorio.uchile.cl/handle/2250/168154}.
      \end{cvitems}
    }
  
  \cventry
    {B.Eng. Mathematical Engineering} % Degree
    {Universidad de Chile} % Institution
    {Santiago, Chile} % Location
    {2012 - 2018} % Date(s)
    {
      \begin{cvitems}
        \item Graduated with highest distinction (7.0/7.0). 
      \end{cvitems}
    }
%---------------------------------------------------------
\end{cventries}


\begin{cvpublication}
	{J1}
	{On the Two-Dimensional Knapsack Problem for Convex Polygons}
	{with Andreas Wiese}
	{TALG}
	{2024}
    {
	\begin{cvitems}
	\item ACM Transactions on Algorithms
	\item[] DOI: \href{https://doi.org/10.1145/3644390}{10.1145/3644390}
	\item Conference version: \hyperlink{paperC2}{[C2]}
	\end{cvitems}
	}
\end{cvpublication}
\begin{cvpublication}
	{J2}
	{On a Combinatorial Generation Problem of Knuth}
	{with Ondřej Mička and Torsten Mütze}
	{SICOMP}
	{2022}
    {
	\begin{cvitems}
	\item SIAM Journal on Computing
	\item[] DOI: \href{https://doi.org/10.1137/20M1377394}{10.1137/20M1377394}
	\item Conference version: \hyperlink{paperC3}{[C3]}
	\end{cvitems}
	}
\end{cvpublication}
\begin{cvpublication}
	{J3}
	{Combinatorial Generation via Permutation Languages. III. Rectangulations}
	{with Torsten Mütze}
	{DCG}
	{2022}
    {
	\begin{cvitems}
	\item Discrete & Computational Geometry
	\item[] DOI: \href{https://doi.org/10.1007/s00454-022-00393-w}{10.1007/s00454-022-00393-w}
	\item Conference version: \hyperlink{paperC4}{[C4]}
	\end{cvitems}
	}
\end{cvpublication}
\begin{cvpublication}
	{J4}
	{Combinatorial Generation via Permutation Languages. IV. Elimination trees}
	{with Jean Cardinal and Torsten Mütze}
	{TALG}
	{2024}
    {
	\begin{cvitems}
	\item To appear in ACM Transactions on Algorithms
	\item[] Available on arXiv:\href{https://arxiv.org/abs/2106.16204}{2106.16204}
	\item Conference version: \hyperlink{paperC5}{[C5]}
	\end{cvitems}
	}
\end{cvpublication}
\begin{cvpublication}
	{J5}
	{Star Transposition Gray Codes for Multiset Permutations}
	{with Petr Gregor and Torsten Mütze}
	{JGT}
	{2023}
    {
	\begin{cvitems}
	\item Journal of Graph Theory
	\item[] DOI: \href{https://doi.org/10.1002/jgt.22915}{10.1002/jgt.22915}
	\item Conference version: \hyperlink{paperC6}{[C6]}
	\end{cvitems}
	}
\end{cvpublication}
\begin{cvpublication}
	{J6}
	{The Hamilton Compression of Highly Symmetric Graphs}
	{with Petr Gregor and Torsten Mütze}
	{AOCO}
	{2023}
    {
	\begin{cvitems}
	\item Annals of Combinatorics
	\item[] DOI: \href{https://doi.org/10.1007/s00026-023-00674-y}{10.1007/s00026-023-00674-y}
	\item Conference version: \hyperlink{paperC8}{[C8]}
	\end{cvitems}
	}
\end{cvpublication}
\begin{cvpublication}
	{J7}
	{Combinatorial Generation via Permutation Languages. V. Acyclic orientations}
	{with Jean Cardinal, Hung P. Hoang, Ondřej Mička, and Torsten Mütze}
	{SIDMA}
	{2023}
    {
	\begin{cvitems}
	\item SIAM Journal on Discrete Mathematics
	\item[] DOI: \href{https://doi.org/10.1137/23M1546567}{10.1137/23M1546567}
	\item Conference version: \hyperlink{paperC9}{[C9]}
	\end{cvitems}
	}
\end{cvpublication}
\begin{cvpublication}
	{J8}
	{Traversing Combinatorial 0/1-Polytopes via Optimization}
	{with Torsten Mütze}
	{SICOMP}
	{2024}
    {
	\begin{cvitems}
	\item SIAM Journal on Computing
	\item[] DOI: \href{https://doi.org/10.1137/23M1612019}{10.1137/23M1612019}
	\item Conference version: \hyperlink{paperC11}{[C11]}
	\end{cvitems}
	}
\end{cvpublication}
\begin{cvpublication}
	{C1}
	{The Minimum Cost Query Problem on Matroids with Uncertainty Areas}
	{with José A. Soto}
	{ICALP}
	{2019}
    {
	\begin{cvitems}
	\item In Proc. 46th International Colloquium on Automata, Languages, and Programming
	\item[] DOI: \href{https://doi.org/10.4230/LIPIcs.ICALP.2019.83}{10.4230/LIPIcs.ICALP.2019.83}
	\item Journal version in preparation
	\end{cvitems}
	}
\end{cvpublication}
\begin{cvpublication}
	{C2}
	{On the Two-Dimensional Knapsack Problem for Convex Polygons}
	{with Andreas Wiese}
	{ICALP}
	{2020}
    {
	\begin{cvitems}
	\item In Proc. 47th International Colloquium on Automata, Languages, and Programming
	\item[] DOI: \href{https://doi.org/10.4230/LIPIcs.ICALP.2020.84}{10.4230/LIPIcs.ICALP.2020.84}
	\item Journal version: \hyperlink{paperC1}{[J1]}
	\end{cvitems}
	}
\end{cvpublication}
\begin{cvpublication}
	{C3}
	{On a Combinatorial Generation Problem of Knuth}
	{with Ondřej Mička and Torsten Mütze}
	{SODA}
	{2021}
    {
	\begin{cvitems}
	\item In Proc. 32nd SIAM Symposium on Discrete Algorithms
	\item[] DOI: \href{https://doi.org/10.5555/3458064.3458110}{10.5555/3458064.3458110}
	\item Journal version: \hyperlink{paperC2}{[J2]}
	\end{cvitems}
	}
\end{cvpublication}
\begin{cvpublication}
	{C4}
	{Efficient Generation of Rectangulations via Permutation Languages}
	{with Torsten Mütze}
	{SoCG}
	{2021}
    {
	\begin{cvitems}
	\item In Proc. 37th Symposium on Computational Geometry
	\item[] DOI: \href{https://doi.org/10.4230/LIPIcs.SoCG.2021.54}{10.4230/LIPIcs.SoCG.2021.54}
	\item Journal version: \hyperlink{paperC3}{[J3]}
	\end{cvitems}
	}
\end{cvpublication}
\begin{cvpublication}
	{C5}
	{Efficient Generation of Elimination Trees and Graph Associahedra}
	{with Jean Cardinal and Torsten Mütze}
	{SODA}
	{2022}
    {
	\begin{cvitems}
	\item In Proc. 33rd SIAM Symposium on Discrete Algorithms
	\item[] DOI: \href{https://doi.org/10.1137/1.9781611977073.84}{10.1137/1.9781611977073.84}
	\item Journal version: \hyperlink{paperC4}{[J4]}
	\end{cvitems}
	}
\end{cvpublication}
\begin{cvpublication}
	{C6}
	{Star Transposition Gray Codes for Multiset Permutations}
	{with Petr Gregor and Torsten Mütze}
	{STACS}
	{2022}
    {
	\begin{cvitems}
	\item In Proc. 39th Symposium on Theoretical Aspects of Computer Science
	\item[] DOI: \href{https://doi.org/10.4230/LIPIcs.STACS.2022.34}{10.4230/LIPIcs.STACS.2022.34}
	\item Journal version: \hyperlink{paperC5}{[J5]}
	\end{cvitems}
	}
\end{cvpublication}
\begin{cvpublication}
	{C7}
	{All Your Base(s) Are Belong to Us: Listing All Bases of a Matroid by Greedy Exchanges}
	{with Torsten Mütze and Aaron Williams}
	{FUN}
	{2022}
    {
	\begin{cvitems}
	\item In Proc. 11th International Conference on Fun with Algorithms
	\item[] DOI: \href{https://doi.org/10.4230/LIPIcs.FUN.2022.22}{10.4230/LIPIcs.FUN.2022.22}
	\item Journal version in preparation
	\end{cvitems}
	}
\end{cvpublication}
\begin{cvpublication}
	{C8}
	{The Hamilton Compression of Highly Symmetric Graphs}
	{with Petr Gregor and Torsten Mütze}
	{MFCS}
	{2022}
    {
	\begin{cvitems}
	\item In Proc. 47th Mathematical Foundations of Computer Science
	\item[] DOI: \href{https://doi.org/10.4230/LIPIcs.MFCS.2022.54}{10.4230/LIPIcs.MFCS.2022.54}
	\item[] MFCS 2022 best paper award
	\item Journal version: \hyperlink{paperC6}{[J6]}
	\end{cvitems}
	}
\end{cvpublication}
\begin{cvpublication}
	{C9}
	{Zigzagging Through Acyclic Orientations of Graphs and Hypergraphs}
	{with Jean Cardinal, Hung P. Hoang, and Torsten Mütze}
	{SODA}
	{2023}
    {
	\begin{cvitems}
	\item In Proc. 34th SIAM Symposium on Discrete Algorithms
	\item[] DOI: \href{https://doi.org/10.1137/1.9781611977554.ch117}{10.1137/1.9781611977554.ch117}
	\item Journal version: \hyperlink{paperC7}{[J7]}
	\end{cvitems}
	}
\end{cvpublication}
\begin{cvpublication}
	{C10}
	{Kneser Graphs are Hamiltonian}
	{with Torsten Mütze and Namrata}
	{STOC}
	{2023}
    {
	\begin{cvitems}
	\item In Proc. 55th ACM Symposium on Theory of Computing
	\item[] DOI: \href{https://doi.org/10.1145/3564246.3585137}{10.1145/3564246.3585137}
	\item Journal version submitted to Advances in Mathematics
	\end{cvitems}
	}
\end{cvpublication}
\begin{cvpublication}
	{C11}
	{Traversing Combinatorial 0/1-Polytopes via Optimization}
	{with Torsten Mütze}
	{FOCS}
	{2023}
    {
	\begin{cvitems}
	\item In Proc. 64th IEEE Symposium on Foundations of Computer Science
	\item[] DOI: \href{https://doi.org/10.1109/FOCS57990.2023.00076}{10.1109/FOCS57990.2023.00076}
	\item Journal version: \hyperlink{paperC8}{[J8]}
	\end{cvitems}
	}
\end{cvpublication}
\begin{cvpublication}
	{C12}
	{On the Hardness of Gray Code Problems for Combinatorial Objects}
	{with Namrata and Aaron Williams}
	{WALCOM}
	{2024}
    {
	\begin{cvitems}
	\item In Proc. 18th Workshop on Algorithms and Computation
	\item[] DOI: \href{https://doi.org/10.1007/978-981-97-0566-5_9}{10.1007/978-981-97-0566-5_9}
	\item Journal version in preparation
	\end{cvitems}
	}
\end{cvpublication}
\begin{cvpublication}
	{C13}
	{Generating All Invertible Matrices by Row Operations}
	{with Petr Gregor, Hung P. Hoang, and Ondřej Mička}
	{ISAAC}
	{2024}
    {
	\begin{cvitems}
	\item To appear in Proc. 35th International Symposium on Algorithms and Computation
	\item[] Available on arXiv:\href{https://arxiv.org/abs/2405.01863}{2405.01863}
	\item Journal version in preparation
	\end{cvitems}
	}
\end{cvpublication}
\begin{cvpublication}
	{C14}
	{Impartial Selection under Combinatorial Constraints}
	{with Javier Cembrano and Max Klimm}
	{WINE}
	{2024}
    {
	\begin{cvitems}
	\item To appear in Proc. 20th Workshop on Internet and Network Economics
	\item[] Available on arXiv:\href{https://arxiv.org/abs/2409.20477}{2409.20477}
	\item Journal version in preparation
	\end{cvitems}
	}
\end{cvpublication}
\begin{cvpublication}
	{P1}
	{Graphs that Admit a Hamiltonian Path are Cup-Stackable}
	{with Petr Gregor, Torsten Mütze, and Francesco Verciani}
	{arXiv}
    {2024}
	{
	\begin{cvitems}
		\item Available on arXiv:\href{https://arxiv.org/abs/2401.06189}{2401.06189}
		\item Submitted to Discrete Mathematics
	\end{cvitems}
	}
\end{cvpublication}
\begin{cvpublication}
	{P2}
	{Set Selection with Uncertain Weights: Non-Adaptive Queries and Thresholds}
	{with Christoph Dürr, José A. Soto, and José Verschae}
	{arXiv}
    {2024}
	{
	\begin{cvitems}
		\item Available on arXiv:\href{https://arxiv.org/abs/2404.17214}{2404.17214}
	\end{cvitems}
	}
\end{cvpublication}
% %-------------------------------------------------------------------------------
%	SECTION TITLE
%-------------------------------------------------------------------------------
\cvsection{Grants \& Awards}

\begin{cvhonors}


 %---------------------------------------------------------


 \cvhonor
 {Richard Rado Prize nominee} % Award
 {Discrete Mathematics of the German Mathemtical Society.} % Event
 {Berlin, Germany} % Location
 {2024} % Date(s)
 {Awarded biyearly to an outstanding dissertation in discrete mathematics.} % Description
%  


 \cvhonor
 {MATH+ Dissertation award} % Award
 {Berlin Mathematical School and Enstein Foundation.} % Event
 {Berlin, Germany} % Location
 {2023} % Date(s)
 {Awarded to at most three math Ph.D. dissertations in Berlin each year.} % Description
%  
%---------------------------------------------------------
  \cvhonor
    {Best paper award} % Award
    {International Symposium on Mathematical Foundations of Computer Science.} % Event
    {Vienna, Austria} % Location
    {2022} % Date(s)
    {Sponsored by the European Association for Theoretical Computer Science.} % Description
% ---------------------------------------------------------
  \cvhonor
    {Becas Chile Ph.D. fellowship} % Award
    {Chilean Science Foundation (ANID).} % Event
    {Chile} % Location
    {2019-2023} % Date(s)
    {Application ranked 6th out of 586 applicants.} % Description
%---------------------------------------------------------
  \cvhonor
    {Outstanding student} % Award
    {Universidad de Chile Engineering School.} % Event
    {Santiago, Chile} % Location
    {2015-2017} % Date(s)
    {Granted to top ~5\% students each year.} % Description
%---------------------------------------------------------
\end{cvhonors}


%-------------------------------------------------------------------------------
%	SECTION TITLE
%-------------------------------------------------------------------------------
\cvsection{Selected Talks}


%-------------------------------------------------------------------------------
%	CONTENT
%-------------------------------------------------------------------------------
\begin{cvtalks}
  \cvtalk
  {Set Selection with Uncertain Weights \vspace{-5mm}}
  {}
  {\begin{cvitems}
      \item \descriptionstyle{AGCO seminar, 2024.} \hfill \cvreferencewho{U. de Chile, Chile}
    \end{cvitems}}
  


  \cvtalk
  {Greedy strategies for combinatorial generation \vspace{-5mm}}
  {based on [\hyperlink{paper8}{8}]}
  {
    \begin{cvitems}
      \item \descriptionstyle{2024 DMV Symposium on Discrete Mathematics.} \hfill \cvreferencewho{Berlin, Germany}
    \end{cvitems}
  }

  \vspace{5mm}
\end{cvtalks}
% %-------------------------------------------------------------------------------
%	SECTION TITLE
%-------------------------------------------------------------------------------
\cvsection{Teaching}


%-------------------------------------------------------------------------------
%	CONTENT
%-------------------------------------------------------------------------------
\cvsubsection{As Main Lecturer\hfill}\vspace{1 mm}

\begin{cvtalks}
  \cvtalk
  {Theory of Algorithms}
  {}
  {
    \begin{cvitems}
      \item \descriptionstyle{Fall 2025.} \hfill \cvreferencewho{U. de O'Higgins, Chile} 
    \end{cvitems}
  }
  \cvtalk
  {Linear Algebra}
  {}
  {
    \begin{cvitems}
      \item \descriptionstyle{Spring 2024, Fall 2025} \hfill \cvreferencewho{U. de O'Higgins, Chile} 
    \end{cvitems}
  }
  \cvtalk
  {Linear Algebra Crash Course}
  {}
  {
    \begin{cvitems}
      \item \descriptionstyle{Summer 2021.} \hfill \cvreferencewho{U. de Chile, Chile} 
    \end{cvitems}
  }


\cvsubsection{As Teaching Assistant\hfill}\vspace{1 mm}

\begin{cvtalks}
  \cvtalk
  {Mixed Linear Programming: Theory and Laboratory}
  {}
  {
    \begin{cvitems}
      \item \descriptionstyle{Fall 2017, Fall 2018.} \hfill \cvreferencewho{U. de Chile, Chile} 
    \end{cvitems}
  }
  \cvtalk
  {Calculability and Computation Complexity}
  {}
  {
    \begin{cvitems}
      \item \descriptionstyle{Fall 2018.} \hfill \cvreferencewho{U. de Chile, Chile} 
    \end{cvitems}
  }  
  \cvtalk
  {Differential and Integral Calculus}
  {}
  {
    \begin{cvitems}
      \item \descriptionstyle{Spring  2017.} \hfill \cvreferencewho{U. de Chile, Chile} 
    \end{cvitems}
  }
  \cvtalk
  {Combinatorial Algorithms}
  {}
  {
    \begin{cvitems}
      \item \descriptionstyle{Spring 2017.} \hfill \cvreferencewho{U. de Chile, Chile} 
    \end{cvitems}
  }
  \cvtalk
  {Introduction to Algebra}
  {}
  {
    \begin{cvitems}
      \item \descriptionstyle{Fall 2015, Spring 2016, Fall 2017.} \hfill \cvreferencewho{U. de Chile, Chile} 
    \end{cvitems}
  }
  \cvtalk
  {Linear Algebra}
  {}
  {
    \begin{cvitems}
      \item \descriptionstyle{Spring 2014, Spring 2015, Fall 2016, Spring 2016.} \hfill \cvreferencewho{U. de Chile, Chile} 
    \end{cvitems}
  }  
  \cvtalk
  {Combinatorics}
  {}
  {
    \begin{cvitems}
      \item \descriptionstyle{Fall 2016.} \hfill \cvreferencewho{U. de Chile, Chile} 
    \end{cvitems}
  }

  
  
\end{cvtalks}

\cvsubsection{As Guest Lecturer\hfill}\vspace{1 mm}


\begin{cvtalks}
  \cvtalk
  {Combinatorial Generation: Graphs, Structures, and Algorithms}
  {}
  {
    \begin{cvitems}
      \item \descriptionstyle{Winter 2022, Winter 2023} \hfill \cvreferencewho{Charles U., Czechia} 
    \end{cvitems}
  }
\end{cvtalks}



\end{cvtalks}
% %-------------------------------------------------------------------------------
%	SECTION TITLE
%-------------------------------------------------------------------------------
\cvsection{Language Skills}
%-------------------------------------------------------------------------------
%	CONTENT
%-------------------------------------------------------------------------------
\begin{cvskills}

%---------------------------------------------------------
  \cvskill
    {Spanish} % Category
    {Native speaker} % Skills

%---------------------------------------------------------
  \cvskill
    {English} % Category
    {Fluent} % Skills

%---------------------------------------------------------
  \cvskill
    {German} % Category
    {Basic} % Skills
    
\end{cvskills}

% %-------------------------------------------------------------------------------
%	SECTION TITLE
%-------------------------------------------------------------------------------
\cvsection{International Conferences Attendance}

FOCS 2023, CORE 2023, FUN2022, SODA2022, CORE 2021, ICALP 2021, PP2021, SoCG 2021, SODA 2021, SAGT 2020, ICALP 2020, IPCO 2020, ICALP 2019.


=======
%-------------------------------------------------------------------------------
%	SECTION TITLE
%-------------------------------------------------------------------------------
\cvsection{Personal Information}
%-------------------------------------------------------------------------------
%	CONTENT
%-------------------------------------------------------------------------------
\begin{cvskills}

%---------------------------------------------------------
  \cvskill
    {Full name} % Category
    {Arturo Ignacio Merino Figueroa} % Skills

%---------------------------------------------------------
  \cvskill
    {D.O.B.} % Category
    {24/08/1993} % Skills

%---------------------------------------------------------
\cvskill
{P.O.B.} % Category
{Ñuñoa, Santiago, Chile} % Skills

%---------------------------------------------------------
  \cvskill
    {Citizenship} % Category
    {Chilean} % Skills
    
\end{cvskills}


%-------------------------------------------------------------------------------
%	SECTION TITLE
%-------------------------------------------------------------------------------
\cvsection{Professional Experience}


%-------------------------------------------------------------------------------
%	CONTENT
%-------------------------------------------------------------------------------
\begin{cventries}

%---------------------------------------------------------
\cventry
{Assistant Professor} % Job title
{Universidad de O'Higgins} % Organization
{Rancagua, Chile} % Location
{2023 - Current} % Date(s)
{
  \begin{cvitems}
    \item Assistant professor at the Engineering Institute of the Universidad de O'Higgins. 
  \end{cvitems}
}

\cventry
{Postdoctoral Researcher} % Job title
{Universität des Saarlandes and Max Planck Institut für Informatik} % Organization
{Saarbrücken, Germany} % Location
{2023-2024} % Date(s)
{
  \begin{cvitems}
    \item Researcher within the algorithms group.
  \end{cvitems}
}
\cventry
    {Project Engineer} % Job title
    {Center for Mathematical Modeling - Resource management laboratory} % Organization
    {Santiago, Chile} % Location
    {2018-2019} % Date(s)
    {
    \begin{cvitems}
      \item Designer and developer of heuristics and algorithms for vehicle routing problems.
    \end{cvitems}
    }
%---------------------------------------------------------
\end{cventries}


%-------------------------------------------------------------------------------
%	SECTION TITLE
%-------------------------------------------------------------------------------
\cvsection{Education}


%-------------------------------------------------------------------------------
%	CONTENT
%-------------------------------------------------------------------------------
\begin{cventries}

  \cventry
    {Ph.D. in Mathematics} % Degree
    {Technische Universität Berlin} % Institution
    {Berlin, Germany} % Location
    {2019 - 2023} % Date(s)
    {%
    \begin{cvitems}
      \item Graduated summa cum laude.
      Supervised by Prof. Torsten Mütze.
      \item[] Thesis title: ``Combinatorial Generation: Greedy Approaches and Symmetry.''
      \item[] Awardee of the 2023 MATH+ Dissertation award.
      \item[] DOI: \href{https://doi.org/10.14279/depositonce-19653}{10.14279/depositonce-19653}.
    \end{cvitems}
}
  
  \cventry
    {M.Eng. in Applied Mathematics} % Degree
    {Universidad de Chile} % Institution
    {Santiago, Chile} % Location
    {2017 - 2018} % Date(s)
    {
      \begin{cvitems}
      \item Graduated with highest distinction (6.8/7.0). Supervised by Prof. José Soto. 
      \item[] Thesis title: ``Optimal Bases of Uncertainty Matroids and How to Compute Them With Queries of Minimum Cost.''
      \item[] Available at Universidad de Chile's repository: \href{https://repositorio.uchile.cl/handle/2250/168154}{repositorio.uchile.cl/handle/2250/168154}.
      \end{cvitems}
    }
  
  \cventry
    {B.Eng. Mathematical Engineering} % Degree
    {Universidad de Chile} % Institution
    {Santiago, Chile} % Location
    {2012 - 2018} % Date(s)
    {
      \begin{cvitems}
        \item Graduated with highest distinction (7.0/7.0). 
      \end{cvitems}
    }
%---------------------------------------------------------
\end{cventries}



\begin{cvpublication}
	{J1}
	{On the Two-Dimensional Knapsack Problem for Convex Polygons}
	{with Andreas Wiese}
	{TALG}
	{2024}
    {
	\begin{cvitems}
	\item ACM Transactions on Algorithms
	\item[] DOI: \href{https://doi.org/10.1145/3644390}{10.1145/3644390}
	\item Conference version: \hyperlink{paperC2}{[C2]}
	\end{cvitems}
	}
\end{cvpublication}
\begin{cvpublication}
	{J2}
	{On a Combinatorial Generation Problem of Knuth}
	{with Ondřej Mička and Torsten Mütze}
	{SICOMP}
	{2022}
    {
	\begin{cvitems}
	\item SIAM Journal on Computing
	\item[] DOI: \href{https://doi.org/10.1137/20M1377394}{10.1137/20M1377394}
	\item Conference version: \hyperlink{paperC3}{[C3]}
	\end{cvitems}
	}
\end{cvpublication}
\begin{cvpublication}
	{J3}
	{Combinatorial Generation via Permutation Languages. III. Rectangulations}
	{with Torsten Mütze}
	{DCG}
	{2022}
    {
	\begin{cvitems}
	\item Discrete & Computational Geometry
	\item[] DOI: \href{https://doi.org/10.1007/s00454-022-00393-w}{10.1007/s00454-022-00393-w}
	\item Conference version: \hyperlink{paperC4}{[C4]}
	\end{cvitems}
	}
\end{cvpublication}
\begin{cvpublication}
	{J4}
	{Combinatorial Generation via Permutation Languages. IV. Elimination trees}
	{with Jean Cardinal and Torsten Mütze}
	{TALG}
	{2024}
    {
	\begin{cvitems}
	\item To appear in ACM Transactions on Algorithms
	\item[] Available on arXiv:\href{https://arxiv.org/abs/2106.16204}{2106.16204}
	\item Conference version: \hyperlink{paperC5}{[C5]}
	\end{cvitems}
	}
\end{cvpublication}
\begin{cvpublication}
	{J5}
	{Star Transposition Gray Codes for Multiset Permutations}
	{with Petr Gregor and Torsten Mütze}
	{JGT}
	{2023}
    {
	\begin{cvitems}
	\item Journal of Graph Theory
	\item[] DOI: \href{https://doi.org/10.1002/jgt.22915}{10.1002/jgt.22915}
	\item Conference version: \hyperlink{paperC6}{[C6]}
	\end{cvitems}
	}
\end{cvpublication}
\begin{cvpublication}
	{J6}
	{The Hamilton Compression of Highly Symmetric Graphs}
	{with Petr Gregor and Torsten Mütze}
	{AOCO}
	{2023}
    {
	\begin{cvitems}
	\item Annals of Combinatorics
	\item[] DOI: \href{https://doi.org/10.1007/s00026-023-00674-y}{10.1007/s00026-023-00674-y}
	\item Conference version: \hyperlink{paperC8}{[C8]}
	\end{cvitems}
	}
\end{cvpublication}
\begin{cvpublication}
	{J7}
	{Combinatorial Generation via Permutation Languages. V. Acyclic orientations}
	{with Jean Cardinal, Hung P. Hoang, Ondřej Mička, and Torsten Mütze}
	{SIDMA}
	{2023}
    {
	\begin{cvitems}
	\item SIAM Journal on Discrete Mathematics
	\item[] DOI: \href{https://doi.org/10.1137/23M1546567}{10.1137/23M1546567}
	\item Conference version: \hyperlink{paperC9}{[C9]}
	\end{cvitems}
	}
\end{cvpublication}
\begin{cvpublication}
	{J8}
	{Traversing Combinatorial 0/1-Polytopes via Optimization}
	{with Torsten Mütze}
	{SICOMP}
	{2024}
    {
	\begin{cvitems}
	\item SIAM Journal on Computing
	\item[] DOI: \href{https://doi.org/10.1137/23M1612019}{10.1137/23M1612019}
	\item Conference version: \hyperlink{paperC11}{[C11]}
	\end{cvitems}
	}
\end{cvpublication}
\begin{cvpublication}
	{C1}
	{The Minimum Cost Query Problem on Matroids with Uncertainty Areas}
	{with José A. Soto}
	{ICALP}
	{2019}
    {
	\begin{cvitems}
	\item In Proc. 46th International Colloquium on Automata, Languages, and Programming
	\item[] DOI: \href{https://doi.org/10.4230/LIPIcs.ICALP.2019.83}{10.4230/LIPIcs.ICALP.2019.83}
	\item Journal version in preparation
	\end{cvitems}
	}
\end{cvpublication}
\begin{cvpublication}
	{C2}
	{On the Two-Dimensional Knapsack Problem for Convex Polygons}
	{with Andreas Wiese}
	{ICALP}
	{2020}
    {
	\begin{cvitems}
	\item In Proc. 47th International Colloquium on Automata, Languages, and Programming
	\item[] DOI: \href{https://doi.org/10.4230/LIPIcs.ICALP.2020.84}{10.4230/LIPIcs.ICALP.2020.84}
	\item Journal version: \hyperlink{paperC1}{[J1]}
	\end{cvitems}
	}
\end{cvpublication}
\begin{cvpublication}
	{C3}
	{On a Combinatorial Generation Problem of Knuth}
	{with Ondřej Mička and Torsten Mütze}
	{SODA}
	{2021}
    {
	\begin{cvitems}
	\item In Proc. 32nd SIAM Symposium on Discrete Algorithms
	\item[] DOI: \href{https://doi.org/10.5555/3458064.3458110}{10.5555/3458064.3458110}
	\item Journal version: \hyperlink{paperC2}{[J2]}
	\end{cvitems}
	}
\end{cvpublication}
\begin{cvpublication}
	{C4}
	{Efficient Generation of Rectangulations via Permutation Languages}
	{with Torsten Mütze}
	{SoCG}
	{2021}
    {
	\begin{cvitems}
	\item In Proc. 37th Symposium on Computational Geometry
	\item[] DOI: \href{https://doi.org/10.4230/LIPIcs.SoCG.2021.54}{10.4230/LIPIcs.SoCG.2021.54}
	\item Journal version: \hyperlink{paperC3}{[J3]}
	\end{cvitems}
	}
\end{cvpublication}
\begin{cvpublication}
	{C5}
	{Efficient Generation of Elimination Trees and Graph Associahedra}
	{with Jean Cardinal and Torsten Mütze}
	{SODA}
	{2022}
    {
	\begin{cvitems}
	\item In Proc. 33rd SIAM Symposium on Discrete Algorithms
	\item[] DOI: \href{https://doi.org/10.1137/1.9781611977073.84}{10.1137/1.9781611977073.84}
	\item Journal version: \hyperlink{paperC4}{[J4]}
	\end{cvitems}
	}
\end{cvpublication}
\begin{cvpublication}
	{C6}
	{Star Transposition Gray Codes for Multiset Permutations}
	{with Petr Gregor and Torsten Mütze}
	{STACS}
	{2022}
    {
	\begin{cvitems}
	\item In Proc. 39th Symposium on Theoretical Aspects of Computer Science
	\item[] DOI: \href{https://doi.org/10.4230/LIPIcs.STACS.2022.34}{10.4230/LIPIcs.STACS.2022.34}
	\item Journal version: \hyperlink{paperC5}{[J5]}
	\end{cvitems}
	}
\end{cvpublication}
\begin{cvpublication}
	{C7}
	{All Your Base(s) Are Belong to Us: Listing All Bases of a Matroid by Greedy Exchanges}
	{with Torsten Mütze and Aaron Williams}
	{FUN}
	{2022}
    {
	\begin{cvitems}
	\item In Proc. 11th International Conference on Fun with Algorithms
	\item[] DOI: \href{https://doi.org/10.4230/LIPIcs.FUN.2022.22}{10.4230/LIPIcs.FUN.2022.22}
	\item Journal version in preparation
	\end{cvitems}
	}
\end{cvpublication}
\begin{cvpublication}
	{C8}
	{The Hamilton Compression of Highly Symmetric Graphs}
	{with Petr Gregor and Torsten Mütze}
	{MFCS}
	{2022}
    {
	\begin{cvitems}
	\item In Proc. 47th Mathematical Foundations of Computer Science
	\item[] DOI: \href{https://doi.org/10.4230/LIPIcs.MFCS.2022.54}{10.4230/LIPIcs.MFCS.2022.54}
	\item[] MFCS 2022 best paper award
	\item Journal version: \hyperlink{paperC6}{[J6]}
	\end{cvitems}
	}
\end{cvpublication}
\begin{cvpublication}
	{C9}
	{Zigzagging Through Acyclic Orientations of Graphs and Hypergraphs}
	{with Jean Cardinal, Hung P. Hoang, and Torsten Mütze}
	{SODA}
	{2023}
    {
	\begin{cvitems}
	\item In Proc. 34th SIAM Symposium on Discrete Algorithms
	\item[] DOI: \href{https://doi.org/10.1137/1.9781611977554.ch117}{10.1137/1.9781611977554.ch117}
	\item Journal version: \hyperlink{paperC7}{[J7]}
	\end{cvitems}
	}
\end{cvpublication}
\begin{cvpublication}
	{C10}
	{Kneser Graphs are Hamiltonian}
	{with Torsten Mütze and Namrata}
	{STOC}
	{2023}
    {
	\begin{cvitems}
	\item In Proc. 55th ACM Symposium on Theory of Computing
	\item[] DOI: \href{https://doi.org/10.1145/3564246.3585137}{10.1145/3564246.3585137}
	\item Journal version submitted to Advances in Mathematics
	\end{cvitems}
	}
\end{cvpublication}
\begin{cvpublication}
	{C11}
	{Traversing Combinatorial 0/1-Polytopes via Optimization}
	{with Torsten Mütze}
	{FOCS}
	{2023}
    {
	\begin{cvitems}
	\item In Proc. 64th IEEE Symposium on Foundations of Computer Science
	\item[] DOI: \href{https://doi.org/10.1109/FOCS57990.2023.00076}{10.1109/FOCS57990.2023.00076}
	\item Journal version: \hyperlink{paperC8}{[J8]}
	\end{cvitems}
	}
\end{cvpublication}
\begin{cvpublication}
	{C12}
	{On the Hardness of Gray Code Problems for Combinatorial Objects}
	{with Namrata and Aaron Williams}
	{WALCOM}
	{2024}
    {
	\begin{cvitems}
	\item In Proc. 18th Workshop on Algorithms and Computation
	\item[] DOI: \href{https://doi.org/10.1007/978-981-97-0566-5_9}{10.1007/978-981-97-0566-5_9}
	\item Journal version in preparation
	\end{cvitems}
	}
\end{cvpublication}
\begin{cvpublication}
	{C13}
	{Generating All Invertible Matrices by Row Operations}
	{with Petr Gregor, Hung P. Hoang, and Ondřej Mička}
	{ISAAC}
	{2024}
    {
	\begin{cvitems}
	\item To appear in Proc. 35th International Symposium on Algorithms and Computation
	\item[] Available on arXiv:\href{https://arxiv.org/abs/2405.01863}{2405.01863}
	\item Journal version in preparation
	\end{cvitems}
	}
\end{cvpublication}
\begin{cvpublication}
	{C14}
	{Impartial Selection under Combinatorial Constraints}
	{with Javier Cembrano and Max Klimm}
	{WINE}
	{2024}
    {
	\begin{cvitems}
	\item To appear in Proc. 20th Workshop on Internet and Network Economics
	\item[] Available on arXiv:\href{https://arxiv.org/abs/2409.20477}{2409.20477}
	\item Journal version in preparation
	\end{cvitems}
	}
\end{cvpublication}
\begin{cvpublication}
	{P1}
	{Graphs that Admit a Hamiltonian Path are Cup-Stackable}
	{with Petr Gregor, Torsten Mütze, and Francesco Verciani}
	{arXiv}
    {2024}
	{
	\begin{cvitems}
		\item Available on arXiv:\href{https://arxiv.org/abs/2401.06189}{2401.06189}
		\item Submitted to Discrete Mathematics
	\end{cvitems}
	}
\end{cvpublication}
\begin{cvpublication}
	{P2}
	{Set Selection with Uncertain Weights: Non-Adaptive Queries and Thresholds}
	{with Christoph Dürr, José A. Soto, and José Verschae}
	{arXiv}
    {2024}
	{
	\begin{cvitems}
		\item Available on arXiv:\href{https://arxiv.org/abs/2404.17214}{2404.17214}
	\end{cvitems}
	}
\end{cvpublication}

%-------------------------------------------------------------------------------
%	SECTION TITLE
%-------------------------------------------------------------------------------
\cvsection{Grants \& Awards}

\begin{cvhonors}


 %---------------------------------------------------------


 \cvhonor
 {Richard Rado Prize nominee} % Award
 {Discrete Mathematics of the German Mathemtical Society.} % Event
 {Berlin, Germany} % Location
 {2024} % Date(s)
 {Awarded biyearly to an outstanding dissertation in discrete mathematics.} % Description
%  


 \cvhonor
 {MATH+ Dissertation award} % Award
 {Berlin Mathematical School and Enstein Foundation.} % Event
 {Berlin, Germany} % Location
 {2023} % Date(s)
 {Awarded to at most three math Ph.D. dissertations in Berlin each year.} % Description
%  
%---------------------------------------------------------
  \cvhonor
    {Best paper award} % Award
    {International Symposium on Mathematical Foundations of Computer Science.} % Event
    {Vienna, Austria} % Location
    {2022} % Date(s)
    {Sponsored by the European Association for Theoretical Computer Science.} % Description
% ---------------------------------------------------------
  \cvhonor
    {Becas Chile Ph.D. fellowship} % Award
    {Chilean Science Foundation (ANID).} % Event
    {Chile} % Location
    {2019-2023} % Date(s)
    {Application ranked 6th out of 586 applicants.} % Description
%---------------------------------------------------------
  \cvhonor
    {Outstanding student} % Award
    {Universidad de Chile Engineering School.} % Event
    {Santiago, Chile} % Location
    {2015-2017} % Date(s)
    {Granted to top ~5\% students each year.} % Description
%---------------------------------------------------------
\end{cvhonors}



%-------------------------------------------------------------------------------
%	SECTION TITLE
%-------------------------------------------------------------------------------
\cvsection{Selected Talks}


%-------------------------------------------------------------------------------
%	CONTENT
%-------------------------------------------------------------------------------
\begin{cvtalks}
  \cvtalk
  {Set Selection with Uncertain Weights}
  {based on [\hyperlink{paper8}{8}]}
  {
    \begin{cvitems}
      \item \descriptionstyle{Optimization Oberseminar, 2024} \hfill \cvreferencewho{RPTU, Germany} 
      %\item[] \url{https://math.rptu.de/ags/opt/lehre/oberseminar\#c31244}
      \item \descriptionstyle{DISCOGA seminar, 2024.} \hfill \cvreferencewho{TU Berlin, Germany}
      %\item[] \url{https://www3.math.tu-berlin.de/coga/study/researchseminar/} 
    \end{cvitems}
  }
  \cvtalk
  {Kneser Graphs are Hamiltonian}
  {based on [\hyperlink{paper8}{8}]}
  {
    \begin{cvitems}
      \item \descriptionstyle{Annual meeting of the German Mathematical Society, 2023.} \hfill \cvreferencewho{Ilmenau, Germany} 
      %\item[] \url{https://www.tu-ilmenau.de/fileadmin/Bereiche/MN/mathematik/DMV_2023/book_of_abstracts_DMV2023.pdf}
      \item \descriptionstyle{Graph Theory Seminar, 2023.} \hfill \cvreferencewho{TU Ilmenau, Germany} 
      %\item[] \url{https://www.tu-ilmenau.de/fileadmin/Bereiche/MN/lnrg/abstract_Arturo_Merino.pdf} 
    \end{cvitems}
  }
  \cvtalk
  {Traversing Combinatorial 0/1-Polytopes via Optimization}
  {based on [\hyperlink{paper11}{11}]}
  {
    \begin{cvitems}
      \item \descriptionstyle{Algorithms Lunch Seminar, 2024.} \hfill \cvreferencewho{U. Libré de Bruxelles, Belgium}
      \item \descriptionstyle{Theory of Computing Seminar, 2023.} \hfill \cvreferencewho{Charles U., Czechia}
      %\item[] \url{https://www.mff.cuni.cz/en/iuuk/events/seminar-on-theory-of-computing}
      \item \descriptionstyle{Computer Science Colloquium, 2023.} \hfill \cvreferencewho{Queens College CUNY, USA}
      %\item[] \url{https://www.cs.qc.cuny.edu/seminar/FA23/index.html}
      \item \descriptionstyle{64th Symposium on the Theory of Computing, 2023.} \hfill \cvreferencewho{Santa Cruz, USA}
      %\item[] \url{https://focs.computer.org/2023/schedule/}
      \item \descriptionstyle{Algorithms and Complexity noon seminar, 2023.} \hfill \cvreferencewho{MPI for Informatics, Germany}
      %\item[] \url{https://domino.mpi-inf.mpg.de/internet/events.nsf/c1256081005596d0c125605d007f8191/db02b38cd3e47049c12589b80039e8bb?OpenDocument}
      \item \descriptionstyle{3rd Workshop on Combinatorial Reconfiguration, 2023.} \hfill \cvreferencewho{U. of Paderborn, Germany}
      %\item[] \url{https://core.dais.is.tohoku.ac.jp/media/files/_u/event/file1/2nz2ptvgks.pdf}
      \item \descriptionstyle{Beyond Permutahedra and Associahedra workshop, 2023.} \hfill \cvreferencewho{Weissensee, Austria} 
      %\item[] \url{https://pagcap.lisn.upsaclay.fr/2022-austria-workshop.html}
      \item \descriptionstyle{DISC seminar, 2022.} \hfill \cvreferencewho{U. Adolfo Ibañez, Chile} 
      %\item[] \url{https://ingenieria.uai.cl/phd/disc/news/}
      \item \descriptionstyle{AGCO seminar, 2022.} \hfill \cvreferencewho{U. de Chile, Chile}
      %\item[] \url{https://sites.google.com/view/agco-seminar-chile/home}
      \item \descriptionstyle{DISCOGA seminar, 2022.} \hfill \cvreferencewho{TU Berlin, Germany}
      %\item[] \url{https://www3.math.tu-berlin.de/coga/study/researchseminar/} 
    \end{cvitems}
  }
  \cvtalk
  {The Hamilton Compression of Highly Symmetric Graphs}
  {based on [\hyperlink{paper8}{8}]}
  {
    \begin{cvitems}
      \item \descriptionstyle{Graph Theory Seminar, 2022.} \hfill \cvreferencewho{U. de Chile, Chile} 
      %\item[] \url{https://www.cmm.uchile.cl/?p=44543}
    \end{cvitems}
  }
  \cvtalk
  {Efficient Generation of Rectangulations via Permutation Languages}
  {based on [\hyperlink{paper4}{4}]}
  {
    \begin{cvitems}
      \item \descriptionstyle{Theory of Combinatorial Algorithms Mittagsseminar, 2022.} \hfill \cvreferencewho{ETH Zürich, Switzerland}
      %\item[] \url{https://ti.inf.ethz.ch/ew/mise/mittagssem.html?action=show&what=abstract&id=76b752cd203c6e4850b4730e859ade099665c0fa}
    \end{cvitems}
  }
  \cvtalk
  {All your Bases are Belong to Us}
  {based on [\hyperlink{paper7}{7}]}
  {
    \begin{cvitems}
      \item \descriptionstyle{11th International Conference on Fun with Algorithms, 2022.} \hfill \cvreferencewho{Favignana, Italy} 
      %\item[] \url{https://sites.google.com/view/fun2022/program?authuser=0}
    \end{cvitems}
  }
  \cvtalk
  {Greedy Generation for Hamilton Paths in Rectangulations, Elimination Trees, and Matroids}
  {based on [\hyperlink{paper4}{4},\hyperlink{paper5}{5},\hyperlink{paper7}{7}]}
  {
    \begin{cvitems}
      \item \descriptionstyle{Institute of Geometry seminar, 2022.} \hfill \cvreferencewho{TU Graz, Austria}
      %\item[] \url{https://www.geometrie.tugraz.at/events/merino2022.pdf}
    \end{cvitems}
  }
  \cvtalk
  {Greedily Generating All Bases of a Matroid by Base Exchanges}
  {based on [\hyperlink{paper7}{7}]}
  {
    \begin{cvitems}
      \item \descriptionstyle{Applied Mathematics Noon Lecture, 2022.} \hfill \cvreferencewho{Charles U., Czechia}
      %\item[] \url{https://www.mff.cuni.cz/en/kam/teaching-and-seminars/noon-lectures/2022/26052022}
    \end{cvitems}
  }
  \cvtalk
  {Efficient Generation of Elimination Trees and Graph Associahedra}
  {based on [\hyperlink{paper5}{5}]}
  {
    \begin{cvitems}
      \item \descriptionstyle{33rd ACM-SIAM Symposium on Discrete Algorithms, 2022.} \hfill \cvreferencewho{Alexandria, USA}
      %\item[] \url{https://meetings.siam.org/sess/dsp_programsess.cfm?SESSIONCODE=73701}
      \item \descriptionstyle{IOL \& DISCOGA seminar, 2021.} \hfill \cvreferencewho{TU Berlin \& ZIB, Germany}
      %\item[] \url{https://www3.math.tu-berlin.de/coga/study/researchseminar/} 
      \item \descriptionstyle{Discrete Mathematics Mittagsseminar, 2021.} \hfill \cvreferencewho{TU Berlin, Germany}
      %\item[] \url{https://page.math.tu-berlin.de/~felsner/DiskreMath/seminar.php}
    \end{cvitems}
  }
  \cvtalk
  {Eficient Generation of Rectangulations and Elimination Trees}
  {based on [\hyperlink{paper4}{4},\hyperlink{paper5}{5}]}
  {
    \begin{cvitems}
      \item \descriptionstyle{Séminaire de Combinatoire du Plateau de Saclay, 2021.} \hfill \cvreferencewho{U. Paris-Saclay, France}
      %\item[] \url{https://www.lix.polytechnique.fr/news/350/view}
    \end{cvitems}
  }
  \cvtalk
  {Eficient Generation of Rectangulations via Permutation Languages}
  {based on [\hyperlink{paper4}{4}]}
  {
    \begin{cvitems}
      \item \descriptionstyle{Applied Mathematics Noon Lecture, 2021.} \hfill \cvreferencewho{Charles U., Czechia}
      %\item[] \url{https://www.mff.cuni.cz/en/kam/teaching-and-seminars/noon-lectures/2021/efficient-generation-of-rectangulations-via-permutation-languages}
      \item \descriptionstyle{Workshop on Combinatorial Reconfiguration, 2021.} \hfill \cvreferencewho{U. of Glasgow, Scotland}
      %\item[] \url{https://core.dais.is.tohoku.ac.jp/media/files/_u/event/file3/zen61tirr.pdf}
      \item \descriptionstyle{37th International Symposium on Computational Geometry, 2021.} \hfill \cvreferencewho{U. of Buffalo, USA}
      %\item[] \url{https://cse.buffalo.edu/socg21/program.html}
      \item \descriptionstyle{IOL \& DISCOGA seminar, 2020.} \hfill \cvreferencewho{TU Berlin, Germany}
      %\item[] \url{https://www3.math.tu-berlin.de/coga/study/researchseminar/} 
    \end{cvitems}
  }
  \cvtalk
  {Pattern-Avoiding Permutations and Rectangulations}
  {based on [\hyperlink{paper4}{4}]}
  {
    \begin{cvitems}
      \item \descriptionstyle{19th International Conference on Permutation Patterns, 2021.} \hfill \cvreferencewho{U. of Strathclyde, Scotland}
      %\item[] \url{https://combinatorics.cis.strath.ac.uk/pp2021/PP2021-Programme.pdf}
    \end{cvitems}
  }
  \cvtalk
  {Greedy Strategies for Exhaustive Generation}
  {survey talk}
  {
    \begin{cvitems}
      \item \descriptionstyle{IOL \& DISCOGA seminar, 2021.} \hfill \cvreferencewho{TU Berlin \& ZIB, Germany}
      %\item[] \url{https://www3.math.tu-berlin.de/coga/study/researchseminar/} 
    \end{cvitems}
  }
  \cvtalk
  {Symmetric Hamilton Cycles on Symmetric Graphs}
  {based on [\hyperlink{paper3}{3}]}
  {
    \begin{cvitems}
      \item \descriptionstyle{Berlin Mathematical School conference, 2021.} \hfill \cvreferencewho{Berlin, Germany}
      %\item[] \url{https://bmsstudconf.github.io/2021/schedule.html} 
      \item \descriptionstyle{Graph Theory seminar, 2020.} \hfill \cvreferencewho{U. de Chile, Chile}
      %\item[] \url{https://www.cmm.uchile.cl/wp-content/uploads/2022/11/Arturo-Merino-09-11-22.pdf}
    \end{cvitems}
  }
  \cvtalk
  {On the Two-Dimensional Knapsack Problem for Convex Polygons}
  {based on [\hyperlink{paper2}{2}]}
  {
    \begin{cvitems}
      \item \descriptionstyle{AGCO seminar (available \href{https://drive.google.com/file/d/1fCyPOUXSjNq0gQHgnS0xwQIUWU-IK3b_/view}{here}), 2020.} \hfill \cvreferencewho{U. de Chile, Chile}
      %\item[] \url{https://sites.google.com/view/agco-seminar-chile/home}
      \item \descriptionstyle{47th International Colloquium on Automata, Languages and Programming (\href{https://youtu.be/0q_GryBSoOM}{youtube}), 2020.} \hfill \cvreferencewho{U. des Saarlandes, Germany}
      %\item[] \url{https://icalp2020.saarland-informatics-campus.de/schedule/index.html}
      \item \descriptionstyle{COGA seminar, 2020.} \hfill \cvreferencewho{TU Berlin, Germany}
      %\item[] \url{https://www3.math.tu-berlin.de/coga/study/researchseminar/} 
    \end{cvitems}
  }
  \cvtalk
  {How to Pack Objects into a Knapsack}
  {based on [\hyperlink{paper2}{2}]}
  {
    \begin{cvitems}
      \item \descriptionstyle{Berlin Mathematical School Conference, 2020.} \hfill \cvreferencewho{Berlin, Germany}
      %\item[] \url{https://bmsstudconf.github.io/2020/schedule.html}
    \end{cvitems}
  }
  % \cvtalk
  % {Combinatorial Gray Codes and Permutations}
  % {survey talk}
  % {
  %   \begin{cvitems}
  %     \item \descriptionstyle{Graph Theory seminar, 2019.} \hfill \cvreferencewho{U. de Chile, Chile}
  %     %\item[] Announcement attached.
  %   \end{cvitems}
  % }
  \cvtalk
  {The Minimum Cost Query Problem on Matroids with Uncertainty Areas}
  {based on [\hyperlink{paper1}{1}]}
  {
    \begin{cvitems}
      \item \descriptionstyle{COGA Seminar, 2019.} \hfill \cvreferencewho{TU Berlin, Germany}
      %\item[] \url{https://www3.math.tu-berlin.de/coga/study/researchseminar/} 
      \item \descriptionstyle{46th International Colloquium on Automata, Languages and Programming, 2019.} \hfill \cvreferencewho{U. of Patras, Greece}
      %\item[] \url{https://icalp2019.upatras.gr/calendar.php}
      \item \descriptionstyle{DISC seminar, 2019.} \hfill \cvreferencewho{U. Adolfo Ibañez, Chile}
      %\item[] {\tiny\url{https://ingenieria.uai.cl/evento/seminario-disc-arbol-de-costo-minimo-bajo-incertidumbre-con-acceso-a-consultor/}}
      \item \descriptionstyle{14th Summer School in Discrete Mathematics, 2019.} \hfill \cvreferencewho{Valparaíso, Chile}
      %\item[] \url{https://eventos.cmm.uchile.cl/discretas2019/programa/}
    \end{cvitems}
  }
\end{cvtalks}

%-------------------------------------------------------------------------------
%	SECTION TITLE
%-------------------------------------------------------------------------------
\cvsection{Teaching}


%-------------------------------------------------------------------------------
%	CONTENT
%-------------------------------------------------------------------------------
\cvsubsection{As Main Lecturer\hfill}\vspace{1 mm}

\begin{cvtalks}
  \cvtalk
  {Theory of Algorithms}
  {}
  {
    \begin{cvitems}
      \item \descriptionstyle{Fall 2025.} \hfill \cvreferencewho{U. de O'Higgins, Chile} 
    \end{cvitems}
  }
  \cvtalk
  {Linear Algebra}
  {}
  {
    \begin{cvitems}
      \item \descriptionstyle{Spring 2024, Fall 2025} \hfill \cvreferencewho{U. de O'Higgins, Chile} 
    \end{cvitems}
  }
  \cvtalk
  {Linear Algebra Crash Course}
  {}
  {
    \begin{cvitems}
      \item \descriptionstyle{Summer 2021.} \hfill \cvreferencewho{U. de Chile, Chile} 
    \end{cvitems}
  }


\cvsubsection{As Teaching Assistant\hfill}\vspace{1 mm}

\begin{cvtalks}
  \cvtalk
  {Mixed Linear Programming: Theory and Laboratory}
  {}
  {
    \begin{cvitems}
      \item \descriptionstyle{Fall 2017, Fall 2018.} \hfill \cvreferencewho{U. de Chile, Chile} 
    \end{cvitems}
  }
  \cvtalk
  {Calculability and Computation Complexity}
  {}
  {
    \begin{cvitems}
      \item \descriptionstyle{Fall 2018.} \hfill \cvreferencewho{U. de Chile, Chile} 
    \end{cvitems}
  }  
  \cvtalk
  {Differential and Integral Calculus}
  {}
  {
    \begin{cvitems}
      \item \descriptionstyle{Spring  2017.} \hfill \cvreferencewho{U. de Chile, Chile} 
    \end{cvitems}
  }
  \cvtalk
  {Combinatorial Algorithms}
  {}
  {
    \begin{cvitems}
      \item \descriptionstyle{Spring 2017.} \hfill \cvreferencewho{U. de Chile, Chile} 
    \end{cvitems}
  }
  \cvtalk
  {Introduction to Algebra}
  {}
  {
    \begin{cvitems}
      \item \descriptionstyle{Fall 2015, Spring 2016, Fall 2017.} \hfill \cvreferencewho{U. de Chile, Chile} 
    \end{cvitems}
  }
  \cvtalk
  {Linear Algebra}
  {}
  {
    \begin{cvitems}
      \item \descriptionstyle{Spring 2014, Spring 2015, Fall 2016, Spring 2016.} \hfill \cvreferencewho{U. de Chile, Chile} 
    \end{cvitems}
  }  
  \cvtalk
  {Combinatorics}
  {}
  {
    \begin{cvitems}
      \item \descriptionstyle{Fall 2016.} \hfill \cvreferencewho{U. de Chile, Chile} 
    \end{cvitems}
  }

  
  
\end{cvtalks}

\cvsubsection{As Guest Lecturer\hfill}\vspace{1 mm}


\begin{cvtalks}
  \cvtalk
  {Combinatorial Generation: Graphs, Structures, and Algorithms}
  {}
  {
    \begin{cvitems}
      \item \descriptionstyle{Winter 2022, Winter 2023} \hfill \cvreferencewho{Charles U., Czechia} 
    \end{cvitems}
  }
\end{cvtalks}



\end{cvtalks}

%-------------------------------------------------------------------------------
%	SECTION TITLE
%-------------------------------------------------------------------------------
\cvsection{Language Skills}
%-------------------------------------------------------------------------------
%	CONTENT
%-------------------------------------------------------------------------------
\begin{cvskills}

%---------------------------------------------------------
  \cvskill
    {Spanish} % Category
    {Native speaker} % Skills

%---------------------------------------------------------
  \cvskill
    {English} % Category
    {Fluent} % Skills

%---------------------------------------------------------
  \cvskill
    {German} % Category
    {Basic} % Skills
    
\end{cvskills}


%-------------------------------------------------------------------------------
%	SECTION TITLE
%-------------------------------------------------------------------------------
\cvsection{International Conferences Attendance}

FOCS 2023, CORE 2023, FUN2022, SODA2022, CORE 2021, ICALP 2021, PP2021, SoCG 2021, SODA 2021, SAGT 2020, ICALP 2020, IPCO 2020, ICALP 2019.



>>>>>>> 9b39848997d7ee301ec5ebb92bd9b56c563c5e51
%-------------------------------------------------------------------------------
%	SECTION TITLE
%-------------------------------------------------------------------------------
\cvsection{Service}

%-------------------------------------------------------------------------------
%	CONTENT
%-------------------------------------------------------------------------------
\cvsubsection{Reviewing for International Conferences}

STACS 2025.
% APPROX2024, WG~2024, ICALP~2024 (x2), FPSAC~2024, SoCG~2024, LATIN~2024, SOSA~2024, SODA~2024, WAOA~2023, ESA~2023, ISAAC 2023, SoCG~2023, IPCO~2023, ESA~2022, ICALP~2022, ESA~2021, LAGOS~2021. 

\cvsubsection{Reviewing for Journals}

Information and Computation.

% ACM Transactions on Algorithms (x2), Theory of Computing Systems, IEICE Transactions on Information and Systems, Electronic Journal of Combinatorics, Theoretical Computer Science, Discrete Applied Mathematics, Graphs and Combinatorics, Annals of Combinatorics.

% \cvsubsection{Reviewing for Grants}

% USACH 2023 Internal Grant.

% \cvsubsection{Organizer}

% SOMACHI 2023 Discrete Math session.


 


% %-------------------------------------------------------------------------------
%	SECTION TITLE
%-------------------------------------------------------------------------------
\cvsection{References}


\begin{cvreferences}

  %---------------------------------------------------------
  \cvreference
  {1} % Number
  {Torsten Mütze} % Name
  {torsten.mutze@warwick.ac.uk, +44-2476522368} % Mail
  {Ph.D. advisor, Assistant Professor at University of Warwick.} % Who is

  \cvreference
  {2} % Number
  {Jean Cardinal} % Name
  {jcardin@ulb.ac.be, +32-26505608} % Mail
  {Co-author, Professor at Université Libre de Bruxelles.} % Who is

  \cvreference
  {3} % Number
  {Karl Bringmann} % Name
  {kbringma@mpi-inf.mpg.de, +49-681-30257330} % Mail
  {Postdoctoral host, Professor at Saarland University.} % Who is

  \cvreference
  {4} % Number
  {José Soto} % Name
  {jsoto@dim.uchile.cl, +56-229784438} % Mail
  {Master's advisor, Associate Professor at Universidad de Chile.} % Who is
  %---------------------------------------------------------
  \end{cvreferences}
  
  

% %-------------------------------------------------------------------------------
%	SECTION TITLE
%-------------------------------------------------------------------------------
\cvsection{Computer Skills}


%-------------------------------------------------------------------------------
%	CONTENT
%-------------------------------------------------------------------------------
\begin{cvskills}

%---------------------------------------------------------
  \cvskill
    {Basic} % Category
    {C++} % Skills

%---------------------------------------------------------
  \cvskill
    {Intermediate} % Category
    {Bash, Cplex, Gurobi, Git, R, SageMath} % Skills

%---------------------------------------------------------
  \cvskill
    {Advanced} % Category
    {Java, Python, \LaTeX} % Skills

%---------------------------------------------------------
\end{cvskills}


% %-------------------------------------------------------------------------------
%	SECTION TITLE
%-------------------------------------------------------------------------------
\cvsection{Coauthor List}

\begin{enumerate}
  \item Jean Cardinal (Université Libre de Bruxelles, Belgium)
  \item Petr Gregor (Charles University Prague, Czechia)
  \item Hung Hoang (ETH Zürich, Switzerland)
  \item Ondřej Mička (Charles University Prague, Czechia)
  \item Torsten Mütze (University of Warwick, UK)
  \item Namrata (University of Warwick, UK)
  \item José Soto (Universidad de Chile, Chile)
  \item Andreas Wiese (TU München, Germany)
  \item Aaron Williams (Williams College, USA)
\end{enumerate}
  
  

% %-------------------------------------------------------------------------------
%	SECTION TITLE
%-------------------------------------------------------------------------------
\cvsection{Contributions}

Three publications are relevant to the thesis [J1], [C5], and [C8].
Additionally, there is a conference version of [J1] ([C3]) that is completely overtaken by [J1].
For all three publications, the applicant contributions consisted, together with the coauthors, of
  \begin{itemize}
    \item development of relevant research questions,
    \item revision of the related literature,
    \item development of methods for tackling the posed questions,
    \item coding of computer experiments to aid current methods,
    \item writing of the obtained results. 
  \end{itemize}


\vspace*{\fill}
%-------------------------------------------------------------------------------
\end{document}
